%--------------------------------------------
\section{Function Calls of an RT step}
%--------------------------------------------



\begin{landscape}
{\footnotesize

\begin{tabular}[l]{%
	>{\raggedright\arraybackslash}p{2.6cm}%
	>{\raggedright\arraybackslash}p{2.8cm}%
	>{\raggedright\arraybackslash}p{7cm}%
	>{\raggedright\arraybackslash}p{7cm}%
}
\textbf{task} & \textbf{task type} & \textbf{task purpose} & \textbf{function calls} \\[.5em]
\hline
\hline
Injection Prep &
	Interacting star and gas particles &
	Gather gas particle neighbour data in preparation for the injection &
	\texttt{runner\_iact\_rt\_inject\_prep} in \verb|src/rt/method/rt_iact.h| \\
\hline
Star Emission Rates &
	Work on individual star particles &
	Prepare everything necessary that needs to be done for radiation sources before the radiation sources and the gas interact. &
	\texttt{rt\_compute\_stellar\_emission\_rate} in \verb|src/rt/method/rt.h| \\
\hline
\hline
RT in &
	Implicit&
  Collect dependencies &
	- \\
\hline
Injection &
	Interacting star and gas particles &
	Distribute the radiation from star particles onto gas particles &
	\verb|runner_iact_rt_inject| in \verb|src/rt/method/rt_iact.h| \\
\hline
Ghost1 &
	Work on individual gas particles &
	Finish up any work that needs to be done for gas particles before the next gas $\leftrightarrow$ gas interaction begins &
	\texttt{rt\_finalise\_injection} in \verb|src/rt/method/rt.h|\\
\hline
Gradient &
	Interacting gas and gas particles &
	Compute necessary gradients of the radiation quantities &
	\verb|rt_gradients_collect| in \verb|src/rt/method/rt_gradients.h| and
	\verb|rt_gradients_nonsym_collect| in \verb|src/rt/method/rt_gradients.h|\\
\hline
Ghost2 &
	Work on individual gas particles &
	Finish up any work that needs to be done for gas particles before the next gas $\leftrightarrow$ gas interaction begins &
	\texttt{rt\_end\_gradient} in \verb|src/rt/method/rt.h|\\
\hline
Transport &
	Interacting gas and gas particles &
	Compute/exchange fluxes of homogenized equation of radiative transfer. &
	\verb|runner_iact_rt_transport| in \verb|src/rt/method/rt_iact.h| and
	\verb|runner_iact_nonsym_rt_transport| in \verb|src/rt/method/rt_iact.h|\\
\hline
Transport out &
	Implicit&
  Collect dependencies &
	- \\
\hline
Thermochemistry &
	Work on individual gas particles &
	Finish up any work that needs to be done for gas particles before the thermochemistry part of the computation can begin. Then do the thermochemistry computation. &
	\texttt{rt\_finalise\_transport} in \verb|src/rt/method/rt.h|,
	\verb|rt_do_thermochemistry| in \verb|src/rt/method/rt_thermochemistry.h|\\
\hline
RT out &
	Implicit&
  Collect dependencies &
	- \\
\hline
\end{tabular}


}
\end{landscape}
\newpage


\todo{check whether this is up to date}






%-------------------------------------------------------------------------
\section{Creating Collisional Ionization Equilibrium Initial Conditions}
%-------------------------------------------------------------------------

\todo{Document this}








%---------------------------------------------
\section{Additional Notes}
%---------------------------------------------

%---------------------------------------------
\subsection{Zero Flux with Nonzero Energy}
%---------------------------------------------

We could encounter cases where we have nonzero radiation energy, but zero 
radiation flux. (E.g. through diffusion or when exception handling unphysical 
scenarios.) For these cases, recall that (index $i$ below is for photon group, 
not particle index!)

\begin{align}
	\DELDT{\F_i} + c^2 \ \nabla \cdot \mathds{P}_i &=
		- \sum\limits_{j}^{\mathrm{HI, HeI, HeII}} n_j \sigma_{i j} c \F_i \\
	\mathds{P}_i &= 
		\mathds{D}_i E_i \\
	\mathds{D}_i &= 
		\frac{1- \chi_i}{2} \mathds{I} + \frac{3 \chi_i - 1}{2} \mathbf{n}_i \otimes \mathbf{n}_i\\
	\mathbf{n}_i &= 
		\frac{\F_i}{|\F_i|} \\
	\chi_i &= 
		\frac{3 + 4 f_i ^2}{5 + 2 \sqrt{4 - 3 f_i^2}} \\
	f_i &= 
		\frac{|\F_i|}{c E_i}
\end{align}

For $\F_i = 0$, we get
\begin{align}
	f_i &= 0 \\
	\chi_i &= \frac{3}{5 + 2 \sqrt{4}} = \frac{1}{3} \\
	\mathds{D}_i &= \frac{1- \frac{1}{3}}{2} \mathds{I} = \frac{1}{3} \mathds{I} \\
	\mathds{P}_i &= \mathds{D}_i E_i =  \frac{1}{3} E_i \mathds{I}
\end{align}

which is the solution of the optically thick limit, where the radiation pressure tensor is isotropic.



%------------------------------------------------
\subsection{RT Related Quantities And Their Units}
%------------------------------------------------

It could be helpful to have the commonly used quantities of RT written down
somewhere along with their units. So here you go.


\begin{align*}
	I_\nu (\x, \mathbf{n}, t) &
			&& \text{Specific intensity}
			&& [I_\nu] = \frac{\text{erg}}{\text{cm}^2 \text{ rad Hz s}} \\
	u_\nu (\x, \mathbf{n}, t) &= \frac{I_\nu}{c}
			&& \text{radiation energy density }
			&& [u_\nu] = \frac{\text{erg}}{\text{cm}^3 \text{ rad Hz}}\\
	E_\nu (\x, t) &= \int_{4 \pi} \frac{I_\nu}{c} \de \Omega
			&& \text{total energy density }
			&& [E_\nu] = \frac{\text{erg}}{\text{cm}^3 \text{ Hz}}\\
	E_{rad} (\x, t) &= \int_{0}^{\infty} E_\nu \de \nu
			&& \text{total integrated energy density }
			&& [E_\nu] = \frac{\text{erg}}{\text{cm}^3}\\
	J_\nu (\x, t) &= \int_{4 \pi} I_\nu \frac{\de \Omega}{4 \pi}
			= \frac{c}{4 \pi} E_\nu
			&& \text{mean radiation specific intensity }
			&& [J_\nu] = \frac{\text{erg}}{\text{cm}^2 \text{ Hz s}}\\
	\F_\nu(\x, t) &= \int_{4 \pi}  I_\nu \mathbf{n} \de \Omega
			&& \text{radiation flux }
			&& [\F_\nu] = \frac{\text{erg}}{\text{cm}^2 \text{ s Hz}}\\
	\mathbf{P}_\nu (\x, t) &= \frac{\F_\nu}{c^2}
			&& \text{radiation momentum density }
			&& [\mathbf{P}_\nu] = \frac{\text{erg}}{ \text{cm}^4 \text{ s}^{-1} \text{ Hz}}\\
	\mathds{P}_\nu (\x, t) &= \int_{4 \pi} \frac{I_\nu}{c} \mathbf{n} \otimes \mathbf{n} \de \Omega
			&& \text{radiation pressure tensor }
			&& [\mathds{P}_\nu ] = \frac{\text{erg}}{\text{cm}^3 \text{ Hz}}
\end{align*}



